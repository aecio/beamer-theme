% Copyright 2012 by Aécio S. R. Santos <aecio.solando@gmail.com>.
%
% In principle, this file can be redistributed and/or modified under
% the terms of the GNU Public License, version 2.
%
% However, this file is supposed to be a template to be modified
% for your own needs. For this reason, if you use this file as a
% template and not specifically distribute it as part of a another
% package/program, I grant the extra permission to freely copy and
% modify this file as you see fit and even to delete this copyright
% notice. 

% Redefins a fonte
\usepackage{helvet}

% Define algumas cores usadas
\definecolor{darkgreen}{rgb}{0,.5,.48}
\definecolor{black}{rgb}{0,0,0}
\definecolor{gray}{rgb}{0.3,0.3,0.3}
\definecolor{lightblue}{rgb}{0.2,0.2,0.7}

% Define texto em alerta
\setbeamercolor{alerted text}{fg=lightblue}

% Define tamanho das fontes
\setbeamerfont{frametitle}{parent=structure,size=\Large}
\setbeamerfont{framesubtitle}{parent=frametitle,size=\footnotesize}
\setbeamerfont{itemize/enumerate body}{size=\fontsize{16pt}{17.6pt}}
\setbeamerfont{itemize/enumerate subbody}{size=\fontsize{14pt}{15,4pt}}
\setbeamerfont{itemize/enumerate subsubbody}{size=\footnotesize}

% Redefine a fonte do titulo da capa como negrito
%\setbeamerfont{title}{size=\Large, series=\bfseries}

% Redefine a cor do titulo da capa
\setbeamercolor{title}{fg=darkgreen}

% Redefine a cor do titulo dos slides
\setbeamercolor{frametitle}{fg=darkgreen,size=20pt}

% Redefine bullets com formato circular
%\useinnertheme[shadow]{rounded}
%\setbeamertemplate{blocks}[rounded][shadow=\beamer@themerounded@shadow]
%\setbeamertemplate{items}[ball]

% Redefine o espaço entre items no ambiente 'itemize'
\newlength{\wideitemsep}
\setlength{\wideitemsep}{\itemsep}
\addtolength{\wideitemsep}{0.2pt}
\let\olditem\item
\renewcommand{\item}{\setlength{\itemsep}{\wideitemsep}\olditem}

% Redefine a largura das margens do texto
\setbeamersize{text margin left=1em,text margin right=1em}

% Define a profundidade de itens do sumário
\setcounter{tocdepth}{2}

% Redefine o estilo do titulo de cada slide
\setbeamertemplate{frametitle} {
  \vspace{0.2cm}
  \ifbeamercolorempty[bg]{frametitle}{}{\nointerlineskip}%
  \begin{beamercolorbox}[]{frametitle}
    \ifbeamercolorempty[bg]{frametitle}{}{\nointerlineskip}%
    \usebeamerfont{frametitle}{%
      \strut\insertframetitle\strut\par%
    }
    {%
      \ifx\insertframesubtitle\@empty%
      \else
	\usebeamerfont{framesubtitle}\usebeamercolor[fg]{framesubtitle}\insertframesubtitle\strut\par
      \fi
      \vspace{-.9cm}%
      {
	\textcolor{gray} {\rule[5pt]{\linewidth}{.5pt}\vspace{-8pt}}
      }
    }%  
    \vskip-0.5ex%
    \if@tempswa\else\vskip-.9cm\fi
  \end{beamercolorbox}%
  \vspace{0.2cm}
}

% Remove barra de navegação
\beamertemplatenavigationsymbolsempty 

% Redefine rodapé para exibir número do slide
\setbeamertemplate{footline}{
  \begin{beamercolorbox}[wd=1\paperwidth,ht=2.25ex,dp=1ex,right]{date in head/foot}%
    %\hfill
    \insertframenumber{}                             % Apenas número do slide atual
    %\insertframenumber{} / \inserttotalframenumber  % Número do slide atual e total de slides
    \hspace{2ex} 
  \end{beamercolorbox}
}